\section{Introduction}
The recently published "Fit for 55" package \cite{european_commission_european_2019} by the European Commission outlines the pathway until 2030 to reduce greenhouse gas emissions by \SI{55}{\%} compared with that in 1990 in the European Union (EU). With an eye on the therein described energy policy recommendations, undisputedly, massive efforts across sectors are necessary to enable a sustainable transformation of the energy system (see also \cite{korkmaz2020comparison}). At the same time, there is a need for energy justice complying with the manner of "no one left behind" \cite{sovacool2019decarbonization}. Against this background, the residential building sector calls for particular attention. There are at least three reasons for this: (i) high shares of fossil fuels in the provision of heating service needs (and increasingly cold services as well), (ii) inefficient ways of delivering the heat demand caused by low standards of both building stock and heating devices, and (iii) complex building ownership structures and the property owner/tenant nexus in rented apartments or dwellings.\vspace{0.5cm}

In fact, buildings are responsible for \SI{40}{\%} of the EU energy consumption and \SI{36}{\%} of the greenhouse gas emissions in $2021$. Moreover, the European Commission states that \SI{75}{\%} of the EU's buildings are energy inefficient. The essential factor to improve these indicators is building retrofitting. Passive renovation measures can already make a significant contribution, as \SI{35}{\%} of the EU's buildings are older than \SI{50}{} years. However, the current renovation rate of \SI{1}{\% \per year} alone will not be sufficient for a deep decarbonization of the European building stock \cite{eurocombuildings2021}. Thus, the share of passive (e.g., building insulation) alongside active renovation (e.g., heating system change) measures needs to be increased rapidly to be compliant with European climate plans such as the abovementioned Fit for 55 package. Indeed, European decarbonization scenarios assume a much higher renovation rate of up to \SI{3}{\%} per year in order to achieve climate neutrality \cite{korkmaz2020comparison}. To increase this rate, most scientific literature findings suggest federal financial incentives since renovation measures do not achieve economic viability under current market environments in the EU (see, e.g., Fina et al. \cite{fina2019profitability}, Weber and Wolff \cite{weber2018energy}, and Kumbaroğlu and Madlener \cite{kumbarouglu2012evaluation}).\vspace{0.5cm}

In the last decades, federal financial incentives have already led to the massive market penetration of renewable energy technologies. For example, in recent years, solar photovoltaic (PV) has flooded the electricity markets driven by feed-in tariff programs \cite{hoppmann2014compulsive}. In addition, significant cost reductions were achieved due to efficiency improvements and economies of scale \cite{haas2011historical}. In principle, there are good reasons to learn from the diffusion pathway of solar PV and related experiences. Nevertheless, two aspects are crucial in this context that have received too little attention in the past. First is that the public monetary diffusion of renewable energy must be accompanied by measures ensuring demand-side energy efficiency and thus energy savings. Recently, Poponi et al. \cite{poponi2021subsidisation} conducted a subsidization cost analysis of solar PV in Italy where they concluded that public monetary support strategies are cost-ineffective policy instruments if energy efficiency investments are ignored. Second is that the support in energy transition must be socially balanced in a society with and without private ownership.\vspace{0.5cm}

The scope of this paper aims at exploring how to deal with one of the "hot potatoes" on the road to a sustainable society: to trigger investments for deep decarbonization of the rented residential building sector in terms of heating system change and passive retrofitting. The focus is put on multi-apartment buildings in urban areas that are often heated by natural gas-based heating systems. Moreover, the frequently occurring ownership structure within the building with a single property owner (building or at least apartment owner) and numerous tenants plays a key role in the analysis as this is a generally crucial relationship. Typically, a property owner is the investment decision-maker in terms of potential (active and passive) renovation measures but is not affected in its decision process by an increasing CO\textsubscript{2} price as the most significant parameter determining deep decarbonization. On the contrary, the tenants are at the mercy of the future CO\textsubscript{2} development and have no decision-making power to counteract it, e.g., by changing the heating system.\vspace{0.5cm}

Against this background, the core objective of this work is to set up a cost-optimal and socially balanced subsidization strategy for a multi-apartment building to trigger investments in a sustainable heat supply. A public authority (governance) incentivizes the replacement of the initial natural gas-based heating system toward a sustainable alternative along with building renovation measures (accompanied by reduced heat demand) by monetary support to the property owner and the tenants. Monetary support can be direct payments in the form of an investment grant for the property owner or a subsidy payment for the tenant. Besides, the property owner can also be indirectly financially supported by allowing a rent adjustment as the building is retrofitted. Social balance is defined at the building level from a monetary perspective using the net present value of the governance's total payments for the building's owner (or apartment's owner) and the tenants.\vspace{0.5cm}

The method applied is the development of a linear optimization model. Thereby, the objective function is to minimize the governance's net present value of monetary support over time. The property owner's and tenants' strategy to minimize individual total costs is considered by tailor-made constraints in the modeling framework. The generalized formulation of the model allows to investigate different building types and categorization (size and number of tenants, building efficiency, initial rent price, etc.). This can be helpful to analyze different building stocks.\vspace{0.5cm}

The numerical example examined is an old multi-apartment building with a single property owner and 30 units (tenants). The partially renovated building is located in an urban area (Vienna, Austria) and initially heated by individual gas heating systems at the unit's level. The decarbonization of the heat supply can be achieved by two different investment options, namely, a connection to the district heating network or an implementation of an air-sourced heat pump system on the building level.\vspace{0.5cm}

The paper is organized as follows. Section \ref{stateoftheart} summarizes the current state-of-the-art in literature and outlines the own contribution of this work beyond existing research. Section \ref{methodology} presents the materials and methods developed in this work, including the mathematical formulation of the model, \added{input data, and} scenarios. Section \ref{results} presents the results of this work, including sensitivity analyses of key determining parameters. Section \ref{conclusions} discusses the results, concludes the work, and outlines possible future research.