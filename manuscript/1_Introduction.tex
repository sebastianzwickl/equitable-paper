\section{Introduction}
The recently published "Fit for 55" package \cite{european_commission_european_2019} by the European Commission outlines the pathway until 2030 to reduce greenhouse gas emissions by \SI{55}{\%} compared to 1990 in the Europe Union (EU). With an eye on the therein described energy policy recommendations, undisputedly, massive efforts across sectors are necessary to enable a sustainable transformation of the energy system (see also in \cite{korkmaz2020comparison}). At the same time, there is a need for energy justice complying with the manner of "no one left behind" \cite{sovacool2019decarbonization}. Against this background, the residential building sector calls for particular attention. There are at least three reasons for this: (i) High shares of fossil fuels in the provision of heat service needs (but also the cold service needs as well), (ii) Inefficient ways of delivering the heat demand caused by low standards in terms of building stock quality and heat generation technologies, (iii) Complex building ownership structures and the fact that many people do not live in a property but in rented apartments or dwellings.\vspace{0.5cm}

In fact, buildings are responsible for \SI{40}{\%} of EU energy consumption and \SI{36}{\%} of the greenhouse gas emissions. Moreover, the European Commission states that \SI{75}{\%} of EU's buildings are energy efficient. The essential factor to improve these indicators is building retrofitting. Passive renovation measures can already make a significant contribution, as \SI{35}{\%} of EU's buildings are older than \SI{50}{} years. However, this alone will not bring the European building stock to deep decarbonization. Rather, it is necessary to increase the current renovation rate of \SI{1}{\% \per year} \cite{eurocombuildings2021}. Thus, the share of passive (e.g., insulation improvements) alongside active renovation (e.g., heating system change) measures needs to be increased rapidly to be compliant with European climate plans such as the abovementioned Fit for 55 package. Indeed, European decarbonization scenarios assume a much higher renovation rate up to \SI{3}{\%} in order to achieve climate neutrality \cite{korkmaz2020comparison}. To increase this rate, most scientific literature findings suggest federal financial incentives since renovation measures do not achieve economic viability under current market environments in the EU (see, e.g., Fina et al. \cite{fina2019profitability}, Weber and Wolff \cite{weber2018energy}, and Kumbaroğlu and Madlener \cite{kumbarouglu2012evaluation}).\vspace{0.5cm}

We have already seen in the last decades how federal financial incentives have led to massive market penetration of renewable energy technologies. For example, solar photovoltaic (PV) has flooded the electricity markets driven by public monetary subsidies such as feed-in tariff programs \cite{hoppmann2014compulsive}. In addition, significant cost reductions were found due to efficiency improvements and economies of scale \cite{haas2011historical}. In principle, there are good reasons to think that one can learn from the diffusion pathway of solar PV and related experiences. Nevertheless, two aspects are crucial in this context that has received too little attention in the past. First that the public monetary diffusion of renewable energy has to be accompanied by measures ensuring energy efficiency. Recently, Poponi et al. \cite{poponi2021subsidisation} conducted a subsidization cost analysis of renewable energy deployment in Italy. Studying the diffusion of solar PV, they concluded that public monetary support strategies are a cost-ineffective policy instrument if energy efficiency investments are ignored. And second, that the support needs to be socially balanced in terms of society with and without private ownership, which is essential for many renewable technology investments, not only in the heating sector.\vspace{0.5cm}

The scope of this paper aims at exploring one "hot potato" of a sustainable society future, namely, the decarbonization of the rented residential building sector in terms of heating system change and passive retrofitting measures. A focus lies on multi-apartment buildings in urban areas that are often heated by natural gas-based heating systems. Moreover, the frequently occurring ownership structure within the building with a single landlord (building owner) and numerous tenants plays a key role in the analysis as this is a generally crucial relationship since typically, a building's landlord is the decision-maker in terms of potential (active and passive) retrofitting measures but is not influenced by an increasing CO\textsubscript{2}, as the key determining parameter of deep decarbonization strategies in its decision process yet. On the contrary, the tenants are impacted significantly by the CO\textsubscript{2} price but without ownership, not able to invest in sustainable heat supply measures independently.\vspace{0.5cm}

Against this background, the core objective of this work is to determine a cost-optimal and socially balanced subsidization strategy for a multi-apartment building to trigger a sustainable heat supply. The governance incentivizes the replacement of the initial natural gas-based heating system toward a sustainable alternative along with building renovation measures to increase efficiency and reduce heat demand by monetary support to the landlord and the tenants. Federal subsidy payments can be direct payments in the form of an investment grant for the landlord or a subsidy payment for the tenant. Besides, the owner can also be indirectly financially supported by allowing a rent adjustment as the building is refurbished. Social balance is defined at the building level from a monetary perspective using the net present value of the governance's subsidy payments for the building's owner and the tenants. This is also associated with a significant increase in the efficiency of the heat supply by passive retrofitting measures.\vspace{0.5cm}

The method applied is the development of a linear optimization model. Thereby, the objective function is to minimize the governance's net present value. The landlord's and tenants' strategy to minimize total costs in considered by tailor-made adapted constraints in the modeling framework. The generalized formulation of the model allows to investigate different building types and categorizes (e.g., size and number of tenants, building's efficiency, initial rent price, etc.) that can be helpful to represent different building stocks.\vspace{0.5cm}

The numerical example analyzed is a multi-apartment old building with a single owner and 30 units or tenants respectively. The partially renovated building is located in an urban area and initially heated by individual gas heating systems at the unit's level. The decarbonization of the heat supply can be achieved by two different options, namely, a connection to the district heating network or an implementation of an air-sourced heat pump system.\vspace{0.5cm}

The paper is organized as follows. Section \ref{stateoftheart} summarizes the current state-of-the-art in research and outlines the present study’s own contribution beyond the existing literature. Section \ref{methodology} presents the materials and methods developed in this work including the mathematical formulation of the model, scenario and numerical example description and model validation. Section \ref{results} presents the results of this work, including sensitivity analyses of key determining parameters. Section \ref{conclusions} discusses the results, concludes the work, and outlines possible future research.