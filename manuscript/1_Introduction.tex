\section{Introduction}
The recently published "Fit for 55" package \cite{european_commission_european_2019} by the European Commission outlines the path until 2030 to reduce greenhouse gas emissions by \SI{55}{\%} compared to 1990 in the Europe Union (EU). Undisputedly, massive efforts across sectors are necessary to enable a sustainable transformation of the energy system \cite{korkmaz2020comparison}. Simultaneously, there is a need for energy justice complying with the manner of "no one left behind" \cite{sovacool2019decarbonization}. Against this background, the residential building sector calls for particular attention. With an eye on the residential building's heat demand, there are at least three reasons for this: Firstly, the high shares of fossil fuels in the provision of heat service needs (but also the cold service needs as well). Secondly, the often very inefficient way of delivering the heat demand is caused by low standards in terms of building stock quality and heat generation technology. And thirdly, the complex ownership structure and the fact that many people do not live in a property but in rented apartments or dwellings.\vspace{0.5cm}

In fact, buildings are responible for \SI{40}{\%} of EU energy consumption and \SI{36}{\%} of the greenhouse gas emissions. Moreover, the European Commission states that \SI{75}{\%} of EU's buildings are energy efficient. The essential factor to improve these indicators is building retrofitting. Passive renovation measures can already make a significant contribution, as \SI{35}{\%} of EU's buildings are older than \SI{50}{} years. However, this alone will not bring the European building stock to deep decarbonization. Rather, it is necessary to increase the current renovation rate of \SI{1}{\% \per year} \cite{eurocombuildings2021}. Thus, the share of passive (e.g., insulation improvements) alongside active renovation (e.g., heating system change) measures needs to be increased rapidly to be compliant with European climate plans such as the abovementioned Fit for 55 package. Indeed, European decarbonization scenarios assume a much higher renovation rate up to \SI{3}{\%} in order to achieve climate neutrality \cite{korkmaz2020comparison}. In order to filling this gap, most of scientific literature findings suggest federal financial incentives since renovation measures do not achive economic viability under current the current market environments in the EU (see, e.g., Fina et al. \cite{fina2019profitability}).\vspace{0.5cm}

We have already seen in the last decades how federal financial incentives have led to a massive market penetration of renewable energy technologies. For example, solar photovoltaic has literally flooded the electricity markets driven by public monetary subsidies such as feed-in tariff programms. In addition, significant cost reduction were found due to efficiency improvements and economies of scale \cite{haas2011historical}. There is much to suggest that we can learn from the diffusion pathway of solar photovoltaics and related experiences. Nevertheless, two aspects are important in this context that have received too little attention in the past. First, that the public monetary diffusion of renewable energy have to be accompanied by measures ensuring the energy efficiency. To given an example here. Recently, Poponi et al. \cite{poponi2021subsidisation} conducted a subsidisation cost analysis of renewable energy deployment in Italy. Studying the diffusion of solar photovoltaics, they concluded that public monetary support strategies are a cost-ineffective policy instrument if energy efficiency investments are ignored. And second, that the support is socially balanced in terms of society with and without private ownership, which is essential for many renewable technology investments in the heating sector.\vspace{0.5cm}

The scope of this paper aims at exploring one "hot potato" of a sustainable society future, namely, the decarbonization of the residential building sector in terms of heating system change and passive retrofitting measures. A focus lies on multi-apartment buildings in urban areas that are often heated by natural gas-based heating systems. Moreover, the frequently occurring ownership structure within the building with a single landlord (building owner) and numerous tenants plays a crucial role in the analysis. The building's landlord is the decision-maker in terms of potential (active and passive) retrofitting measures but is not influenced by an increasing CO\textsubscript{2}, as the key determining parameter of deep decarbonization strategies in its decision process yet. On the contrary, the tenants are impacted significantly by the CO\textsubscript{2} but without ownership not able to invest into sustainable heat supply measures.\vspace{0.5cm}

Against this background, the core objective of this work is to determine a cost-optimal and socially balanced governance's subsidization strategy for a multi-apartment building leading to building renovation measures to reduce and decarbonize heat supply. Federal subsidy payments can be direct payments in the form of an investment grant for the owner or a heating cost subsidy for the tenant. Besides, the owner can also be indirectly financially supported by allowing a rent adjustment as the building is refurbished. Social balance is defined at the building level from a monetary perspective using the net present value of the governance's subsidy payments for the building's owner and the tenants. This is also associated with a significant increase in the provision of heat service needs by (i) a heating system change from a fossil-fuel-based technology to a sustainable alternative and (ii) passive retrofitting measures (e.g., insulation improvements). Latter is comprehensively investigated within a sensitivity analysis using the CO\textsubscript{2} price-related energy cost allocation between the landlord ant the tenants as a parameter.\vspace{0.5cm}

The method applied is the development of a linear optimization model. Thereby, the objective function is to minimize the governance's net present value. The landlord's and tenants' strategy to minimize total costs in considered by tailor-made adapted constraints in the modeling framework. The generalized formulation of the model allows to investigate different building types and categorizes (e.g., size and number of tenants, building's efficiency, initial rent price, etc.) that can be helpful to represent different building stocks.\vspace{0.5cm}

The numerical example analyzed is a multi-apartment old building with a single owner and 30 dwellings or tenants respectively. The partially renovated building is located in an urban area and initially heated by individual gas heating systems at the dwelling's level. The decarbonization of the heat supply can be achieved by two different options, namely, a connection to the district heating network or an implementation of an air-sourced heat pump system.\vspace{0.5cm}

The paper is organized as follows. Section \ref{stateoftheart} summarizes the current state-of-the-art in research and outlines the present study’s own contribution beyond the existing literature. Section \ref{methodology} presents the materials and methods developed in this work including the mathematical formulation of the model, scenario and numerical example description and model validation. Section \ref{results} presents the results of this work, including sensitivity analyses of key determining parameters. Section \ref{conclusions} discusses the results, concludes the work, and outlines possible future research.