\newpage
\section{State-of-the-art and progress beyond}\label{stateoftheart}
This section aims to provide an overview of relevant scientific contributions with respect to this paper's scope. Explicitly not part of the literature review is the already widely discussed topic of sharing renewable energy generation and related peer-to-peer innovations in the light of energy communities. A general study comprehensively dealing with the sharing economy is provided by Codagnone and Martens \cite{codagnone2016scoping}. The reviews from Sousa et al. \cite{sousa2019peer} and Koirala et al. \cite{koirala2016energetic} go into even more depth and with resepect to peer-to-peer energy sharing and energy communities. Also the authors' literature review of this paper in \cite{zwickl2021open} provides a comprehensive review of energy sharing on the local level.\vspace{0.5cm}

Against this background, the focus here lies on three different dimensions without claiming to be absolutely complete in each case. The first dimension is the decarbonization of heating and cooling systems from a system analysis perspective and is described in Section \ref{aspect1}. The second dimension deals with the increasingly importance of justice in the energy system transition and is presented in Section \ref{aspect2}. The third dimension is dedicated to the trade-offs analysis of investment decisions into renewable energy technologies including realted contracting business cases and is discussed in Section \ref{aspect3}. The choice of these focal points, as well as the explicit exclusion of the mentioned topics, are deliberately chosen in order to reflect the DNA of the analysis. 

\subsection{Decarbonizing the provision of heating service needs}\label{aspect1}
The insights obtained from various scientific studies allow us to see the big picture of a decarbonized heating and cooling sector. A fundamental change of the energy carrier mix, alongside a significant efficiency increase, is necessary for a sustainable heating and cooling service need supply. For example, Connolly et al. \cite{connolly2014heat} provide in their study such a strategy and present a decarbonization roadmap for the European heating sector. They propose a new sustainable heat strategy that is based on changes on the demand-side and supply-side. In addition to significant heat savings, integrating sustainable heat sources into centralized heat networks (or district heating networks) and electrifying heat supply (e.g., heat pump) are suggested to achieve a low-carbon heating sector. Seyboth et al. \cite{seyboth2008recognising} focus in their study on supportive energy policy recommendations to enhance the deployment of renewable energy heating and cooling technologies. In particular, this means the integration of renewable sources such as solar, geothermal, and biomass into heating and cooling systems.\vspace{0.5cm}

In general, the sustainable heat source or heat generation technology that is ultimately implemented/used at the end-user levels depends on a number of factors. Among these, geographical and spatial characteristics (e.g., availability of heat network infrastructure, building construction features, outdoor temperature, etc.), in particular, play a crucial role. Su et al. \cite{su2018heating} deal in their study with optimal sustainable heating system alternatives with a special focus on local geographical features of the application site. Their results show that there might not be a one-fits-all solution if decarbonizing local heating systems. However, certain trends are very much emerging in their findings, which can also be confirmed by further case studies. Renewable-fed district heating networks have significant potential to supply heat demand in urban areas. This is exemplarily also shown by the results of Popovski et al. \cite{popovski2018technical}. They state that from a socio-economic perspective, district heating networks with excess heat are the most favorable supply option in densely populated areas. Lake et al. \cite{lake2017review} present a comprehensive review of district heating and cooling systems. They analyze among others the economic feasibility and system identification based on primary energy sources of centralized heating and cooling networks. Rama et al. \cite{rama2018introduction} study the optimal combination of different sustainable heating alternatives.In particular, they show how heat pumps and solarthermal can assist district heating networks. There exist also other alternatives. Sopha et al. \cite{sopha2011exploring} focus in their study on the potential of wood-pellet in Norway, a country with high shares of district heating-based heat supply. They use an agent-based model to identify energy policy options supporting the uptake of such sustainable heating systems. The authors conclude that a stable financial support (i.e., stable wood-pellet price) has the highest impact on the transition of wood-pellet. We refer to Section \ref{aspect3} in this context for a detailed discussion of financial incentives for renewable energy technologies in the heating sector.\vspace{0.5cm}

In any case, there is a need for sustainable alternatives to district heating. Either to complement existing district heating networks in a high-efficient way (e.g., \cite{rama2018introduction} and \cite{sopha2011exploring}) and/or because to compensate non-existing networks in the future. Popovski et al. \cite{popovski2018technical} identify the electrification of the heat supply using heat pumps with photovoltaics as the most cost-competitive alternative from a socio-economic perspective. Leibowicz et al. \cite{leibowicz2018optimal} also show end-use electrification as an optimal strategy for the decarbonization of the heating sector. However, the authors state that the electrification using heat pumps for example only makes sense in combination with building thermal efficiency improvements.\vspace{0.5cm}

In order to emphasize the importance of building renovation measures, we dedicate this concluding paragraph the corresponding literature. In particular, we select papers focusing on the impact of different retorfitting measures on sustainable heating system alternatives. However, we do not differentiate here in detail between different types of retrofitting measures (e.g., purely passive, passive, active, etc.) and refer in this context to the comprehensive literature review of Fina et al. in \cite{fina2019profitability}. Ma et al. \cite{ma2012existing} provide an extensive literature and state-of-the-art analysis of retrofitting focusing on existing buildings. Vieites et al. \cite{vieites2015european} elaborate in this context of European initiatives improving  the energy efficiency in existing and old (historic) buildings. Recently, Weinberger et al. \cite{weinberger2021investigating} investigate the impact of retrofitting on district heating networks. Fina et al. \cite{fina2019profitability} put their focus on the profitability of retrofitting of multi-apartment buildings with special consideration of different heating systems. They thoroughly study the implementation of the combination of building-attached/integrated photovoltaics supporting sustainable heating systems. Their results show how (passive) retrofitting measures result in a reduction of the required installed heating system capacity. However, the energy cost reduction achieved from higher building standards are not able to compensate the initial passive renovation investment costs. They conclude that latter significantly depend on the development of the CO\textsubscript{2} price and the assumptions of end-user investment grants as well as subsidies. We again take up these findings associated with financial support in Section \ref{aspect3}

\subsection{Justice in energy systems: fair and socially balanced sustainable energy transition}\label{aspect2}
The issue of justice in energy systems is addressed in various studies. According to them, a key part of achieving climate targets is to ensure that no one is left behind in the climate action. More generally, the three energy justice tenets are distributive, recognition, and procedural\footnote{In some works, restorative and cosmopolitan justice are also mentioned in this context. See, exemplarily in \cite{oxfordjustice2021}.}. Recently, these are comprehensively discussed and reviewed by Pellegrini et al. \cite{pellegrini2020energy}. Considering this work's scope, we put our focus on procedural justice, as it represents measures that reduce potential barriers to new clean energy investments \cite{oxfordjustice2021}.\vspace{0.5cm}

Generally speaking, dealing with just sustainable energy systems is a monumental task and seems to be very challenging to be generalized. However, studies focusing on certain local regions are likely to be the most promising approach. Recently, van Bommel and Höffken conducted a review study focusing on energy justice at the European community level \cite{van2021energy}. Besides that, Lacey-Barnacle et al. \cite{lacey2020energy} focus in their study on energy justice in developing countries. Coming back to this paper's content and spatial scope, Mundaca et al. \cite{mundaca2018successful} propose two local European case studies in Germany and Denmark investigating local energy transition from an energy justice perspective. Their findings are in line with those from Jenkins et al. \cite{jenkins2018humanizing} showing that energy justice and transitions framework can be combined and achieved simultaneously. However, Hiteva and Soacool \cite{hiteva2017harnessing} conclude from a business model perspective that energy justice may be realized through market principles but not through the market alone. We continue discussing this point in Section \ref{aspect3} when dealing with necessary (financial) incentives that foster the sustainable energy transition. \vspace{0.5cm} 

Recently, Hanke et al. \cite{hanke2021renewable} investigate renewable energy communities and their capability to deliver energy justice. They explore insights from 71 European cases and highlight the necessity of distributing affordable energy to vulnerable households. Furthermore, it is necessary to focus in this regard on low-income households. Exemplarily, Xu and Chen \cite{xu2019energy} propose on the basis of their generated results that low-income households need tailored assistance to ensure energy justice. In particular, they demonstrate that low-income households are renters and thus have fewer energy efficiency appliances. Sovacool et al. \cite{sovacool2019temporality} heat in the same direction and discuss the special difficulties for households without the capital for sustainable energy investments and for those that do not own their own home such as renters. Moreover, renters also often have higher residential heating energy use intensity, an energy efficiency proxy \cite{reames2016targeting}. In this context, Greene \cite{greene2011uncertainty} discussed the so-called “efficiency gap” or “energy paradox". He showed that consumers have a bias leading to undervaluation of future energy savings in relation to their expected value. The main reasons are a combination of two aspects, namely, an uncertainty regarding the net value of future fuel savings and the loss aversion of typical consumers. Filling the abovementioned efficiency gap is crucial in order to achieve both the energy transition and energy justice. Sovacool et al. \cite{sovacool2019decarbonization} show that unfolding the energy transition result in deeper injustices investigating four different low-carbon transitions.

\subsection{Overnight investments versus net present value}\label{aspect3}
In particular, this concluding section is about looking at different renewable energy promotion instruments focusing on the heating sector. However, in some places, we refer to literature that deals in detail with the electricity sector. We consider this to be useful for the reader, to show the parallels and differences between the two sectors through comparison. Connor et al. \cite{connor2013devising} provide a fundamental review paper investigating a wide range of policy options that can support the deployment of renewable heat technologies. Masini and Menichetti \cite{masini2012impact} state that despite numerous energy policies implemented to promote renewable energy technologies, the penetration of these remains below expectations. They identify as one main key a lack of appropriate financing investment incentives. Public (financial) incentives are often seen as the most appropriate and efficient measures to fill this gap. Reuter et al. \cite{reuter2012renewable} compare different policy instruments, ranging from feed-in tariffs to investment subsidies, tax credits, portfolio requirements, and certificate systems. While focusing on companies and their willingness for renewable energy technology investments in the electricity sector, they conclude that feed-in tariffs are an effective means promoting these investments\footnote{Zhou et al. \cite{zhou2011designing} provide a study dealing with the effectiveness of public financial incentives. The authors define effectiveness/efficiency as the amount of intervention (e.g., taxes collected, subsidies paid, etc.) to achieve a policy goal. Here, it is essentially the electricity sector that is being studied.}. Similar results also can be found in the study from Couture and Gagnon \cite{couture2010analysis}. Nevertheless, the two latter studies only investigate the deployment of renewable energy technologies in the electricity sector and not in the heating sector.\vspace{0.5cm} 

Building on these literature findings, however, it is of particular importance to differentiate between renewable energy technology investments from companies and private end-customers and households. In contrast to companies, private households are incentivized more effectively by investment grants to invest in renewable energy technologies \cite{roth2020impact}. This distinction and targeted adjustment of public financial incentives are important since private investment is a key driver of the diffusion of renewable energy technologies \cite{wustenhagen2012strategic}. {\O}stergaard et al. \cite{ostergaard2019costs} investigate the investment costs of households to prepare existing buildings for high-efficient and sustainable heating systems. Their results show that customer investments require financial incentives and are required to be motivated economically\footnote{In particular, {\O}stergaard et al. \cite{ostergaard2019costs} show that the investment into an expansion of an existing low-temperature district heating network can be seen significantly differently. For example, a heat supply company achieves economic viability with the investment considering the potential of newly supplied heat demand in the area. However, it is not guaranteed that new consumers aim to be connected to the network since their investment profitability is highly uncertain due to high connection costs and low heat energy price savings.}. In this context, the role of an increasing CO\textsubscript{2} price should also be interpreted with particular circumspection. Although, in general, the literature sees carbon pricing as the most important measure speeding up the sustainable energy system transition (see, for example, Nägeli et al. \cite{nageli2020policies} focusing on the impact of carbon pricing on the residential building sector). However, this does not solve the inherent problem of differential ownership in the residential sector (i.e., landlords and tenants/renters). It is, therefore, only logical that Hecher et al. \cite{hecher2017trigger} focus in their work on the decision-making processes regarding sustainable heating system investments of homeowners. Therefore, there is a gap in the literature dedicated precisely to a heating system change in the residential sector, not neglecting the different ownerships.\vspace{0.5cm} 

We conclude this section with the topic of energy and heat contracting business models and explicitly aim to give only a small overview, as contracting business models themselves are not part of the paper's main scope. A comparative review of municipal energy business models in different countries is given by Brinker and Satchwell \cite{brinker2020comparative}. Kindström and Ottosson \cite{kindstrom2016local} analyze local and regional energy companies offering energy services and conclude that most many of these are experiencing difficulties on the market. One reason is, as Fine et al. state, that the contracting framework itself decreases the economic viability since the contractor business companies (third party) aim to gain profit (i.e., contractor's interest rate). Suhonen and Okkonen \cite{suhonen2013energy} conduct an analysis of energy service companies in the residential heating sector and show a wide-ranging set of barriers of such business models. Moreover, the results of their Finnish case study reveal that this kind of contracting business model is unattractive and not profitable. Brown \cite{brown2018business} investigates business models for residential retrofit in the United Kingdom and the European Union. Fina et al. \cite{fina2020profitability} study the profitability of contracting business cases for shared photovoltaic generation and renovation measures in a residential multi-apartment building. Their results indicate that the profitability of (passive) building renovation measures significantly depends on the carbon price. However, the difficulties of high carbon prices are already addressed above and in the novelties of this work in the next section. Furthermore, Fina et al. focus explicitly in their study on building owners and neglect different ownership relationships.  

\subsection{Progress beyond state-of-the-art}\label{novelties}
Based on the abovementioned literature review, the scientific contribution and the novelties of this paper can be summarized as follows:
\begin{itemize}
	\item A sustainable heating system change with a focus on the efficient provision of heat service needs at the multi-apartment building is carried out, emphasizing the ownership structure of the building and related financial interests of the building owner and the different tenants/renters. In particular, this addresses one of the hot potatoes of deep decarbonization strategies, namely, the residential heating sector with tenants who cannot change the heating system on their own due to missing ownership, and therefore extra attention need. The sustainable heating system alternative is financially incentivized by a federal subsidy strategy considering monetary justice between the landlord and the tenants. 
	\item The developed analytical framework determines a cost-optimal and socially balanced subsidization strategy from the governance incentivzing a just heating system decarbonization at the building level. Especially, the optimization model allows a quantitative analysis of justice in low-carbon residential heating sector including the agent's specific monetary interests and the ownership structure within a building. Thus, this work focus on the trade-off analysis between the governance, landlord and tenants.
	\item The sensitivity analysis of sharing carbon-related energy costs between the building owner and tenants on the one hand, and linking the governance's strategy directly to building retrofitting measures, on the other hand, represent substantial innovations in the field of research for decarbonizing the existing building stock with landlord and tenants relations. The obtained insights can help build a more reliable understanding of decarbonizing the existing (rented) building stock. Even more, this work may contribute to rapidly increasing the renovation rate, often seen as a key for a high-efficient and decarbonized residential sector. 
\end{itemize}