\newpage
\section{State-of-the-art and progress beyond}\label{stateoftheart}
\definecolor{col}{HTML}{0F00FF}
%This indicates the necessity of conducting comprehensive re-
%search in this field. 
\begin{itemize}
	\item[\textcolor{col}{\textbullet}] \textcolor{col}{Wir fokussieren in unserer Literaturübersicht auf drei wesentliche Aspekte ohne dabei Anspruch auf Vollständigkeit zu erheben. Focus here on renewable technologies in the heating sector}
	\item[\textcolor{col}{\textbullet}] \textcolor{col}{(1) Wie kann aus Sicht einer Systemanalyse die Dekarbonisierung des Wärmesektors umgesetzt werden. Welche erneuerbaren Energie spielen dabei eine wesentliche Rolle} 
	\item[\textcolor{col}{\textbullet}] \textcolor{col}{(2) Wie kann socially-balance erreicht werden im Zusammenhang mit der Penetration von erneuerbaren Energien}
	\item[\textcolor{col}{\textbullet}] \textcolor{col}{(3) Wie kann die Investition wirtschaftlich attratktiv gestaltet werden, Instrumente und welche Barrieren: Trade-offs between overnight investments and net present value decisions}
	\item[\textcolor{col}{\textbullet}] \textcolor{col}{(4) explizit nicht Teil der Literaturübersicht ist: energy communities, sharing renewable energy generation, electricity battery storage, willingness to pay, etc. nur kurz bei justice chapter}
\end{itemize}
\todo{Fehlt noch etwas explizit zu erwähnen, was nicht behandelt wird?}

\subsection{Decarbonizing the provision of heating service needs}
\begin{itemize}
	\item[\textcolor{col}{\textbullet}] \textcolor{col}{Large-scale decarbonization roadmap of the heating sector - "Was ist das große ganze"}
	\item[\textcolor{col}{\textbullet}] \textcolor{col}{Renewable energy heating and cooling - "Welche Quellen haben wir grundsätzlich"}
	\item[\textcolor{col}{\textbullet}] \textcolor{col}{Geograpical characteristics - "Nicht jede Lösung ist überall optimal"}
	\item[\textcolor{col}{\textbullet}] \textcolor{col}{Case study analyses: Fernwärme in urban areas aber wenn nicht vorhanden dann Wärmepumpen}
	\item[\textcolor{col}{\textbullet}] \textcolor{col}{Fernwärme review paper}
	\item[\textcolor{col}{\textbullet}] \textcolor{col}{Bei Fernwärme und insbesondere Wärmepumpen müssen wir über Effizienz reden - "Gebäudestandards"}
	\item[\textcolor{col}{\textbullet}] \textcolor{col}{"Retroffitting absatz in Bezug auf Heizsystem"}
	\item[\textcolor{col}{\textbullet}] \textcolor{col}{"Muss gemacht werden aber oft nicht wirtschaftlich ohne finanziellen Anreize (verweis auf nächstes und übernächstes Kapitel)"}
\end{itemize}
\todo{Ist hier ein roter Faden erkennbar?}

Connolly et al. (2014) \cite{connolly2014heat} A new heat strategy based on district heating and individual heat pumps is designed for the EU27.

Seyboth et al. (2008) \cite{seyboth2008recognising} analyse current policies and experiences and makes recommendations to support enhanced market deployment of REHC technologies to provide greater energy supply security and climate change mitigation.solar, geothermal and biomass resources.

Su et al. (2018) \cite{su2018heating} geographische Eigenschaften sehr wichtig, beim der Wahl der optimale Lösung für das Heizsystem, district heating für urban areas.
Deswegen gibt es eine vielzahl von lokalen Studien, die für einen bestimmten Ort/Charakteristic die optimale Lösung bestimmt. 

Popovski et al. (2018) \cite{popovski2018technical} From a socio-economic perspective, DHC with excess heat is the most feasible solution Heat pumps with photovoltaics are cost-competitive from a socio-economic perspective. Policies are required to support RES from a private-economic perspective.
 
Lake et al. (2017) \cite{lake2017review} Review of district heating and cooling systems for a sustainable future, economic feasibility, system identification based on primary energy sources, including the deployment of more and more renewable energy streams. 
Leibowicz et al. \cite{leibowicz2018optimal}: Consider scenarios including carbon policy and building thermal efficiency improvement. Optimal strategy features end-use electrification and power sector decarbonization. Building thermal efficiency improvements lower the cost of reducing carbon emissions.

\begin{itemize}
	\item Ma et al. (2012) \cite{ma2012existing} retrofitting of existing buildings
	\item Vieites et al. (2015) \cite{vieites2015european} European Initiatives Towards Improving the Energy Efficiency in Existing and old (historic) Buildings
	\item Weinberger et al. (2021) \cite{weinberger2021investigating} Welche Auswirkungen hat es auf das Heizsystem - hier speziell auf District heating
	\item Fina et al. (2019) \cite{fina2019profitability} Profitability of active retrofitting of multi-apartment buildings: Building-attached/integrated photovoltaics with special consideration of different heating systems, energy cost reductions achieved by better building quality cannot compensate for the initial renovation costs (passive retrofitting). hängt sehr stark von dem CO2 Preis ab und welche Investitionsförderung angenommen ist, subsidies sind nicht angenommen hier.
\end{itemize}
%gehen hier nicht ins detail für unterschiedliche retrofitting measures oder ähnliches und verweisen auf die literaturübersicht von \cite{fina2019profitability}, die sehr genau zwischen passiven, aktiven Purely passive energy efficiency measures (e.g. insulation, win-
%dow replacement) Passive measures with an additional solar thermal system for
%hot water and solar space heating energy services; A combination of passive and active (e.g. building attached/integrated PV, battery storage, heating system change,
%HVAC) retrofitting measures.


Rama et al. (2018) \cite{rama2018introduction} combination of different sustainable heating alternatives. In particular, wie Wärmepumpen und Solarthermie Fernwärme assistieren können und hier speziell wenn Gas in die Fernwärme einspeist. 


Exploring policy options for a transition to sustainable heating system diffusion using an agent-based simulation
The present paper aims to identify potential interventions for the uptake of wood-pellet heating in Norway using an agent-based model (ABM).
stable financial support, i.e., a stable wood-pellet price \cite{sopha2011exploring}



\subsection{Justice in energy systems: fair and socially balanced sustainable energy transition}
\begin{itemize}
	\item[\textcolor{col}{\textbullet}] \textcolor{col}{Erklärung, was ist energy justice überhaupt + Review paper}
	\item[\textcolor{col}{\textbullet}] \textcolor{col}{Allgemein sehr schwer, weil geographische Charakteristic berücksichtigt werden muss, western welt nicht auf die gesamte im allgemeinen}
	\item[\textcolor{col}{\textbullet}] \textcolor{col}{Sieht man davon ab und beschränkt sich auf westliche welt, unterschiedliche besitzverhältnisse bezüglich Wärmesektor}	
	\item[\textcolor{col}{\textbullet}] \textcolor{col}{Erste Versuche, das auf lokaler Ebene anzugehen}
	\item[\textcolor{col}{\textbullet}] \textcolor{col}{Beispiele von Studies anführen und Study ob Energy Communities justice sind - social welfare Theresia's paper}
	\item[\textcolor{col}{\textbullet}] \textcolor{col}{Und frage stellen, ist es gerecht wenn nichts passiert oder transition scheitert}
	\item[\textcolor{col}{\textbullet}] \textcolor{col}{müssen speziell auf low-income households schauen und haben oft höhere residential heating energy use intensity im zusammenhang mit efficiency improvements /energy demand reduction wichtig}
\end{itemize}


the three energy justice tenets (distributive, recognitional and procedural)


A review Energy justice - equality \cite{pellegrini2020energy}
Sehr schwer aber zu generalisieren weil geopgraphische und lokale Apsekte berücksichtigt werden sollten
Energy justice in the transition to low carbon energy systems: Exploring key themes in interdisciplinary research: non-western 
Westliche Welt nicht auf die ganze Welt schließen (inklusive nicht entwickelte Länder) \cite{broto2018energy} case study von Mozambik

Harnessing social innovation for energy justice: A business model perspective: Energy justice may be operationalised through market principles but not through the market alone \cite{hiteva2017harnessing}

We examine two local energy transitions from an energy justice perspective \cite{mundaca2018successful}

show that energy justice and transitions frameworks can be combined \cite{jenkins2018humanizing}

Recently 2021, do renewable energy communities deliver energy justice? Exploring insights from 71 European cases \cite{hanke2021renewable}

\begin{itemize}
	\item "We aim to show how when low-carbon transitions unfold, deeper injustices related to equity, distribution, and fairness invariably arise" \cite{sovacool2019decarbonization}
	\item energy justice in household low carbon innovations; low carbon heating; collection of opportunities but also threats; who wins and who loses; difficulties to people without the capital, or who do not own their own home. (\cite{sovacool2019temporality})
\end{itemize}

Wir müssen speziell auf low-income households und renters schauen: Energy efficiency and energy justice for U.S. low-income households: An analysis of multifaceted challenges and potential: Low-income households and renters have fewer energy efficiency appliances; need tailored assistance \cite{xu2019energy}

low-income haben auch höheren residential heating energy use intensity, an energy efficiency proxy \cite{reames2016targeting}

“They are grinding us into the ground” – The lived experience of (in)energy justice amongst low-income older households: Energy justice was experienced on four separately distinguishable levels of social relationships: intra-households, household-energy retailer relations, immediate social networks and wider social relations. simple retrofits improved householder heating capabilities \cite{willand2018they}


\subsection{Trade-offs between overnight investments and net present value decisions}

\begin{itemize}
	
	\item[\textcolor{col}{\textbullet}] \textcolor{col}{Review paper welche Kriterien/Aspekte beeinflussen Investition in erneuerbare Energien}
	\item[\textcolor{col}{\textbullet}] \textcolor{col}{Finanzielle Unsicherheit der Hauptgrund warum nicht stärker investiert wird}
	\item[\textcolor{col}{\textbullet}] \textcolor{col}{neben investoren private investitionen sehr wichtig, damit gemeint auf lokaler ebene small-scale}
	\item[\textcolor{col}{\textbullet}] \textcolor{col}{Problem zum Beispiel Fernwärmeausbau in Gebiet profitabel für Unternehmen, aber nicht für Endkunden beispielsweise.}
	\item[\textcolor{col}{\textbullet}] \textcolor{col}{die meisten Arbeiten sprechen davon, dass öffentliche Anreize notwendig sind.}
	\item[\textcolor{col}{\textbullet}] \textcolor{col}{Effizienz wird so bestimmt, dass möglichst wenig eingegriffen werden soll}
	\item[\textcolor{col}{\textbullet}] \textcolor{col}{Einspeisetarif aber das bei Wärmesektor schwer}
	\item[\textcolor{col}{\textbullet}] \textcolor{col}{Studien Wärmesektor homeowners’ decision-making processes, 1-Familien und 2-Familien homeowners’ decision-making processes aber nicht Besitzverhältnisse auf Mehrparteienhäuser}
\end{itemize}

\begin{itemize}
	\item Ozorhon et al. (2018) \cite{ozorhon2018generating} Literaturübersicht welche Kriterien Investitionsentscheidung in erneuerbare Energien Technologien am stärksten beeinflussen. Allerdings wird allgemein von Investoren gesprochen und nicht explizit auf den privaten Sektor bzw. Staat fokussiert. 
	% main criteria that influence the decisions of the investors are determined based on an extensive literature survey and investigation of sector practice. 	
	\item Masini et al. (2012) \cite{masini2012impact} finanzielle Anreize sind der Hauptgrund warum erneuerbare Energien nicht stärker umgesetzt werden.	
	%	Yet, despite their appeal, and the numerous policies implemented to promote these technologies, the diffusion of RE projects remains somehow below expectations. This limited penetration is also due to a lack of appropriate financing and to a certain reluctance to invest in these technologies.
	\item Reuter et al. (2012) \cite{reuter2012renewable} allgemeinen überblick welche öffentlichen Anreize für Investitionen in erneuerbare Energie	
	% Einspeisetariffen hin zu Investitionszuschüssen, Steuergutschriften, Portfolioanforderungen und Zertifikatssystem	(Public incentives for companies to invest in renewable technologies range from feed-in tariffs, to investment subsidies, tax credits, portfolio requirements and certificate systems.) compare different policies to foster investment into renewables
	\item Zhou et al. (2011) \cite{zhou2011designing} effizienz einer Maßnahme nach dem notwendigen Maß an Eingriff
	% whereas the efficiency is defined as the amount of intervention, including taxes collected, subsidies paid, and GEP cost increase, to achieve the policy goal. The less intervention needed to achieve a goal, the higher the efficiency. proposed bilevel optimization model is to achieve a policy goal with a minimal amount of policy intervention  a centralized planner makes investment decisions for the energy system to serve projected demand of electricity.  berücksichtigt damit nicht die unterschiedlichen Besitzverhältnisse und Bedürfnisse, wir werden daher nicht die kostenminimale Lösung finden
	\item Couture et al. (2010) \cite{couture2010analysis}: bestätigt, dass feed-in tariffs are the most effective policy to encourage the rapid and sustained deployment of renewable energy. Stromsektor aber nicht Wärmesektor/Gebäudesektor
	% an overview of seven different ways to structure the remuneration of a FIT policy, divided into two broad categories: those in which remuneration is dependent on the electricity price, and those that remain independent from it.
	\item Hecher et al. (2017) \cite{hecher2017trigger} fokussiert auf homeowners’ decision-making processes, single and double-family houses. subsidies for heating system tabinvestments and infrastructural adjustments reveal to be most effective for homeowners in problem situations to foster alternative heating systems.
\end{itemize}

\begin{itemize}
	\item Wustenhagen et al. (2012) \cite{wustenhagen2012strategic} Wir brauchen private Investitionen in erneuerbare Energien
	%	Substantial private investment is needed if public policy objectives to increase the share of renewable energy and prevent dangerous anthropogenic climate change are to be achieved. 
	\item Eitan et al. (2019) \cite{eitan2019community} Review,  Beschreibt das so-called "phenomenon" of community and private sector renewable energy partnerships
	%	This review brings to the fore the fast-growing and significant phenomenon of community and private sector renewable energy partnerships, which constitute a fundamental building block of the global renewable energy transformation 
	\item Aslani et al. (2012) \cite{aslani2012prime} Stellt auf Basis einer umfassenden Literaturübersicht fest, dass private Sektor und dessen Investitionen sehr wichtig sind, fokussiert dabei sehr stark auf Stromsektor und kommt zu dem Schluss, dass der Staat als stärkster Treiber gesehen werden kann.
	%	This study will enrich the existing literature on renewable energy policy which emphasises the importance of engaging the private sector. investment in the Middle East 
	\item Rodriguez et al. (2015) \cite{rodriguez2015renewable} Effekte des Staates auf private Investitionen in erneuerbare Energien. Wieder Strom
	%	This paper analyses the effect of government policies and other determinants on private finance investment in renewable energy. A unique dataset of financial transactions for renewable energy projects is constructed using the Bloomberg New Energy Finance database. The dataset covers 87 countries, six renewable energy sectors (wind, solar, biomass, small hydropower, marine and geothermal) and the 2000–2011 time-span. In a first set of models undertaken at the level of the financial deal we find that, in contrast to quota-based schemes, price-based support schemes are positively correlated with private finance contributions.
	\item Schmidt et al. (2013) \cite{schmidt2013attracting} Selbst wenn der Wärmesektor untersucht wird, geht es oftmals um elektrifizierung
	% Attracting private investments into rural electrification — A case study on renewable energy based village grids in Indonesia private sector are essential to scale-up the diffusion. However, the findings also point to the need for government action in order to further improve the risk/return profile and thereby attract private investments for RVGs. 
	\item Williams et al. (2015) \cite{williams2015enabling} Barrieren für Teilnahme des privaten Sektors an dezentraler Elektrifizierungsprojekten
	%	The purpose of this paper is to review barriers to private sector participation in decentralized electrification projects and to identify solutions that have been implemented and proposed to overcome these barriers. The barriers discussed include unsecure revenue streams, inability to finance projects, and long-term project risks such as grid encroachment. The range of interventions and business models reviewed include methods to secure reliable demand, subsidy models, risk guarantees, and different revenue models. 
	\item Ostergaard et al. \cite{ostergaard2019costs} Investments in heating systems are attractive from an energy system perspective, Customer investments in heating systems should be motivated economically.
	\item Nageli et al. (2020) \cite{nageli2020policies} increase in the CO2 tax as well as subsidies are effective in speeding up the transition in the beginning, aber eben auch hier nicht die unterschiedlichen Besitzverhältnisse 
\end{itemize}

\subsection{Progress beyond state-of-the-art}
\begin{itemize}
	\item heating system change and sustainable heat supply takes place
	\item analytical and modeling framework; justice; qualitative analysis
	\item Trade-off analysis between governance, landlords, and tenants $\longrightarrow$ required incentives
	\item Sensitivity analysis helps us to bettter understand the different ownership structure and its influence on the sustainable energy transition
\end{itemize}