\section{State-of-the-art and progress beyond}\label{stateoftheart}
This section aims to provide an overview of relevant scientific contributions with respect to this paper's scope. The focus here lies on three different dimensions. The first dimension covers the decarbonization of heating and cooling systems from a system analysis perspective (see Section \ref{aspect1}). The second dimension deals with the increasingly importance of justice in the energy system transition (see Section \ref{aspect2}). The third dimension is dedicated to the trade-offs analysis of investment decisions into renewable energy technologies including contracting business cases (see Section \ref{aspect3}). The choice of these focal points are deliberately chosen in order to reflect the DNA of the analysis. Intentionally not part of the literature review (out of scope of this paper's analysis) is the already widely discussed topic of sharing renewable energy generation and related peer-to-peer innovations in the light of energy communities\footnote{A general study comprehensively dealing with the sharing economy is provided by Codagnone and Martens \cite{codagnone2016scoping}. The reviews from Sousa et al. \cite{sousa2019peer} and Koirala et al. \cite{koirala2016energetic} go into even more depth with resepect to peer-to-peer energy sharing and energy communities. Also the authors' literature review of the paper in \cite{zwickl2021open} provides a comprehensive review of energy sharing on the local level. The recently published review papers Cabeza et al. \cite{cabeza2018integration} and Zhang et al. \cite{zhang2019review} collect a variety of contributions focusing on similar topics acknowledged above.}.

\subsection{Decarbonizing heating and cooling service needs}\label{aspect1}
The insights obtained from various scientific studies discloses the big picture of a decarbonized heating and cooling sector which requires a fundamental change of the energy carrier mix, alongside a significant energy efficiency increase. For example, Connolly et al. \cite{connolly2014heat} present a corresponding decarbonization roadmap for the European heating sector. Their sustainable heat strategy proposes changes on both demand-side and supply-side. In addition to significant heat savings, the integration of sustainable heat sources into centralized heat (or district heating) networks and the electrification of heat supply (e.g., heat pump) are proposed to achieve a low-carbon heat sector. Seyboth et al. \cite{seyboth2008recognising} focus in their study on supportive energy policy recommendations to enhance the deployment of renewable energy heating and cooling technologies. In particular, this means integrating of renewable sources such as solar, geothermal, and biomass into heating and cooling systems.\vspace{0.5cm}

In general, the sustainable heat source or heat technology that is ultimately used at the end-user levels depends on a number of factors. Among these, geographical and spatial characteristics (e.g., availability of heat network infrastructure, building construction features, outdoor temperature, etc.) play a crucial role. In this context, Su et al. \cite{su2018heating} focus on local geographical features of the application site. They conclude that there might not be a one-fits-all solution when decarbonizing local heating systems. However, certain trends are very much emerging in their findings, which can also be confirmed by further studies, e.g., that renewable-fed district heating networks have significant potential to supply heat demand in urban areas. Exemplarily, this is shown by the results of Popovski et al. \cite{popovski2018technical}. Lake et al. \cite{lake2017review} present a comprehensive review of district heating and cooling systems with special consideration of the economic feasibility based on primary energy sources. Rama et al. \cite{rama2018introduction} study the optimal combination of different sustainable heating alternatives. In particular, they show how heat pumps and solarthermal can assist district heating networks. Sopha et al. \cite{sopha2011exploring} focus in their study on the potential of wood-pellet in Norway, a country with high shares of district heating-based heat supply. The authors conclude that a stable financial support (i.e., stable wood-pellet price) has the highest impact on the transition of wood-pellet. A continuation of the discussion on financial incentives for renewable energy technologies in the heating sector is conducted in Section \ref{aspect3}.\vspace{0.5cm}

In any case, there are local circumstances where district heating does non fit. Sustainable alternatives must be sought. Either to complement existing district heating networks in a high-efficient way (e.g., \cite{rama2018introduction} and \cite{sopha2011exploring}) and/or to compensate non-existing networks. Popovski et al. \cite{popovski2018technical} identify the electrification of the heat supply using heat pumps with photovoltaics as the most cost-competitive alternative from a socio-economic perspective. Leibowicz et al. \cite{leibowicz2018optimal} also show end-use electrification as an optimal strategy for the decarbonization of the heating sector. However, the authors state that the electrification of the heat sector is only meaningful in combination with building thermal efficiency improvements. Particularly, Kamel et al. review solar systems and their integration with heat pumps \cite{kamel2015solar}.\vspace{0.5cm}

In order to emphasize the importance of building renovation in combination with heating system exchange, this paragraph is dedicated to the corresponding literature. In general, we do not differentiate here in detail between different types of retrofitting measures (e.g., purely passive, passive, active) and refer in this context to the comprehensive literature review of Fina et al. in \cite{fina2019profitability}. Ma et al. \cite{ma2012existing} provide an extensive literature and state-of-the-art analysis of retrofitting focusing on existing buildings. Vieites et al. \cite{vieites2015european} elaborate in this context of European initiatives improving  the energy efficiency in existing and old (historic) buildings. Recently, Weinberger et al. \cite{weinberger2021investigating} investigate the impact of retrofitting on district heating network design. Fina et al. \cite{fina2019profitability} put their focus on the profitability of retrofitting of multi-apartment buildings with special consideration of different heating systems. They thoroughly study the implementation of the combination of building-attached/integrated photovoltaics supporting sustainable heating systems. Their results show how (passive) retrofitting measures result in a reduction of both optimal installed heating system and solar PV capacity. However, the energy cost reduction achieved from higher building standards are not sufficient to compensate the initial passive renovation investment costs. They conclude that economic viability significantly depends on the development of the CO\textsubscript{2} price and end-user investment grants for building renovation.

\subsection{Justice in energy systems: socially balanced sustainable energy transition}\label{aspect2}
The aspect of justice in energy systems is addressed in various studies. According to them, a key part of achieving climate targets is to ensure that no one is left behind in the climate action. More generally, the three energy justice tenets are distributional, recognition, and procedural\footnote{In some works, restorative and cosmopolitan justice are also mentioned in this context, see, exemplarily in \cite{oxfordjustice2021}.}. Recently, they are comprehensively discussed and reviewed by Pellegrini et al. \cite{pellegrini2020energy}. Considering this work's scope, we put our focus on procedural justice, as it represents measures that reduce potential barriers to new clean energy investments \cite{oxfordjustice2021}.\vspace{0.5cm}

Generally speaking, dealing with just sustainable energy systems is a monumental task and seems to be very challenging to be generalized. However, studies focusing on certain local areas are likely to be the most promising approach. Recently, van Bommel and Höffken conducted a review study focusing on energy justice at the European community level \cite{van2021energy}. Besides that, Lacey-Barnacle et al. \cite{lacey2020energy} elaborate on energy justice in developing countries. Coming back to this paper's content and spatial scope, Mundaca et al. \cite{mundaca2018successful} present two local European case studies in Germany and Denmark assessing local energy transition from an energy justice perspective. Their findings are in line with those from Jenkins et al. \cite{jenkins2018humanizing} showing that energy justice and transition frameworks can be combined and achieved simultaneously. However, Hiteva and Soacool \cite{hiteva2017harnessing} conclude from a business model perspective that energy justice may be realized through market principles but not through the market alone. We continue discussing this point in Section \ref{aspect3} when dealing with necessary (financial) incentives that foster the sustainable energy transition. \vspace{0.5cm} 

Recently, Hanke et al. \cite{hanke2021renewable} have investigated renewable energy communities and their capability to deliver energy justice. They explore insights from 71 European cases and highlight the necessity of distributing affordable energy to vulnerable households. Furthermore, it is necessary to focus in this regard on low-income households. Exemplarily, Xu and Chen \cite{xu2019energy} propose on the basis of their results that low-income households need tailored assistance to ensure energy justice. In particular, they demonstrate that low-income households are renters and thus have less energy efficient appliances. Sovacool et al. \cite{sovacool2019temporality} point in the same direction and discuss the difficulties for households who lack the capital for sustainable energy investments and predominantly tenants and not owners of their homes. Moreover, renters also often have higher residential heating energy consumption, an energy efficiency indicator \cite{reames2016targeting}. In this context, Greene \cite{greene2011uncertainty} discussed the so-called “efficiency gap” or “energy paradox". He showed that consumers have a bias leading to undervaluation of future energy savings in relation to their expected value. The main reasons are a combination of two aspects, namely, an uncertainty regarding the net value of future fuel savings and the loss aversion of typical consumers. Filling the abovementioned efficiency gap is crucial in order to achieve both the energy transition and energy justice. Sovacool et al. \cite{sovacool2019decarbonization} show that unfolding the energy transition result in deeper injustices.

\subsection{Energy policy instruments}\label{aspect3}
In particular, the following section is about different renewable energy policy instruments supporting on the heating sector. However, in some places, we refer to literature that deals in detail with the electricity sector. We consider this to be useful for the reader, to show the similarities and differences between the two sectors. Connor et al. \cite{connor2013devising} provide a fundamental review paper investigating a wide range of policy options that can support the deployment of renewable heat technologies. Masini and Menichetti \cite{masini2012impact} state that despite numerous energy policies implemented to promote renewable energy technologies, the penetration of these remains below expectations. They identify as one main key a lack of appropriate financing investment incentives. Public (financial) incentives are seen as the most appropriate measures to fill this gap. Reuter et al. \cite{reuter2012renewable} compare different policy instruments, ranging from feed-in tariffs to investment subsidies, tax credits, portfolio requirements, and certificate systems. While focusing on companies and their willingness for renewable energy technology investments in the electricity sector, they conclude that feed-in tariffs are an effective means promoting these investments\footnote{Zhou et al. \cite{zhou2011designing} provide a study dealing with the effectiveness of public financial incentives. The authors define effectiveness/efficiency as the amount of intervention (e.g., taxes collected, subsidies paid, etc.) to achieve a policy goal. Here, it is essentially the electricity sector that is being studied.}. Similar results also can be found in the study from Couture and Gagnon \cite{couture2010analysis}. Nevertheless, the two latter studies only investigate the deployment of renewable energy technologies in the electricity sector and not in the heating sector.\vspace{0.5cm} 

Building on these literature findings, however, it is of particular importance to differentiate between renewable energy technology investments from companies and private households. In contrast to companies, private households are incentivized more effectively by investment grants to invest in renewable energy technologies \cite{roth2020impact}. This distinction and targeted adjustment of public financial incentives is important since private investments are key drivers of the diffusion of renewable energy technologies \cite{wustenhagen2012strategic}. {\O}stergaard et al. \cite{ostergaard2019costs} conclude that the investment costs of households to prepare existing buildings for high-efficient and sustainable heating systems to be designated economically\footnote{In particular, {\O}stergaard et al. \cite{ostergaard2019costs} show that the investment into an expansion of an existing low-temperature district heating network can be seen significantly differently. For example, a heat supply company achieves economic viability with the investment considering the potential of newly supplied heat demand in the area. However, it is not guaranteed that new consumers aim to be connected to the network since their investment profitability is highly uncertain due to high connection costs and low heat energy price savings.}. In this context, the role of an increasing CO\textsubscript{2} price should also be interpreted with particular circumspection. Although, in general, the literature sees carbon pricing as the most important measure speeding up the sustainable energy system transition (see, for example, Nägeli et al. \cite{nageli2020policies} focusing on the impact of carbon pricing on the residential building sector). However, this does not solve the inherent problem of differential ownership in the residential sector (i.e., landlords and tenants/renters). It is, therefore obvious that Hecher et al. \cite{hecher2017trigger} focus in their work on the decision-making processes regarding sustainable heating system investments of homeowners. The ownership structure is often neglected in the literature and insufficiently considered.\vspace{0.5cm} 

Eventually, energy and heat contracting business models tangent this work's scope. However, we explicitly aim to give only a small overview, as contracting business models themselves do not constitute the core of the analysis in this paper. A comparative review of municipal energy business models in different countries is given by Brinker and Satchwell \cite{brinker2020comparative}. Kindström and Ottosson \cite{kindstrom2016local} as well as Fine et al. \cite{fina2020profitability} conclude little optimistic that the contracting framework itself decreases the economic viability since the contractor business companies (third party) aim to gain profit. Suhonen and Okkonen \cite{suhonen2013energy} conduct an analysis of energy service companies in the residential heating sector and show a wide-ranging set of barriers of such business models responsible for non-profitability of contracting business models. Brown \cite{brown2018business} investigates business models for residential retrofit in the United Kingdom and the European Union.

\subsection{Progress beyond state-of-the-art}\label{novelties}
Based on the literature review, the scientific contribution and the novelties of this paper can be summarized as follows:
\begin{itemize}
	\item An equitable and socially balanced change at a currently gas-based towards a sustainable alternative heating system of a rented multi-apartment old building is modeled considering the complex ownership structure and relations between landlord and tenant to "take action".
	\item First and foremost, the governance's aim is that the local heat system transformation takes place. Particularly, the governance incentivizes the sustainable investment through monetary and regulative support for both the landlord and tenants while considering their individual financial interests. The governance's optimal financial support strategy plays a crucial role where as optimality is defined with respect to the high-efficient provision of the residential heat service needs, heat demand reduction, and building efficiency improvements.
	\item The developed analytical framework determines a cost-optimal and socially balanced governance’s subsidization strategy for the decarbonization of the heat demand at the building level. That includes, among others, the profit-oriented behavior of the landlord and the tenants, as well as the abovementioned financial support parity among both sides.  Especially the proposed optimization model allows a detailed quantitative analysis of justice in a low-carbon residential building and heating sector with an eye on the complex ownership structure within buildings. Moreover, this work focuses on the trade-offs between different agents in the energy transition, particularly the government’s role in triggering private sustainable investment decisions and social balance with an eye on the costs of inaction (opportunity costs) and increasing carbon pricing.	
	\item Different sensitivity analyses play a key role in this paper. Insights to the allocation of the costs of inaction among the governance, the landlord, and the tenants can be seen as one of the main novelties of this work. Moreover, the importance of building stock renovation in the context of public subsidy payments is comprehensively discussed. In this context, the obtained insights can help build a more reliable understanding of a sustainable future urban society that does not live in ownership but in highly efficient supplied rented apartments.
\end{itemize}