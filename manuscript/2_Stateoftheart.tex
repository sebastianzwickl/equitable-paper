\newpage
\section{State-of-the-art and progress beyond}\label{stateoftheart}
This section aims to provide an overview of relevant scientific contributions with respect to this paper's scope. Explicitly not part of the literature review is the already widely discussed topic of sharing renewable energy generation and related peer-to-peer innovations in the light of energy communities. A general study comprehensively dealing with the sharing economy is provided by Codagnone and Martens \cite{codagnone2016scoping}. The reviews from Sousa et al. \cite{sousa2019peer} and Koirala et al. \cite{koirala2016energetic} go into even more depth and with resepect to peer-to-peer energy sharing and energy communities. Also the authors' literature review of this paper in \cite{zwickl2021open} provides a comprehensive review of energy sharing on the local level.\vspace{0.5cm}

Against this background, the focus here lies on three different dimensions without claiming to be absolutely complete in each case. The first dimension is the decarbonization of heating and cooling systems from a system analysis perspective and is described in Section \ref{aspect1}. The second dimension deals with the increasingly importance of justice in the energy system transition and is presented in Section \ref{aspect2}. The third dimension is dedicated to the trade-offs analysis of investment decisions into renewable energy technologies including realted contracting business cases and is discussed in Section \ref{aspect3}. The choice of these focal points, as well as the explicit exclusion of the mentioned topics, are deliberately chosen in order to reflect the DNA of the analysis. 

\subsection{Decarbonizing the provision of heating service needs}\label{aspect1}
The insights obtained from various scientific studies allow us to see the big picture of a decarbonized heating and cooling sector. A fundamental change of the energy carrier mix, alongside a significant efficiency increase, is necessary for a sustainable heating and cooling service need supply. For example, Connolly et al. \cite{connolly2014heat} provide in their study such a strategy and present a decarbonization roadmap for the European heating sector. They propose a new sustainable heat strategy that is based on changes on the demand-side and supply-side. In addition to significant heat savings, integrating sustainable heat sources into centralized heat networks (or district heating networks) and electrifying heat supply (e.g., heat pump) are suggested to achieve a low-carbon heating sector. Seyboth et al. \cite{seyboth2008recognising} focus in their study on supportive energy policy recommendations to enhance the deployment of renewable energy heating and cooling technologies. In particular, this means the integration of renewable sources such as solar, geothermal, and biomass into heating and cooling systems.\vspace{0.5cm}

In general, the sustainable heat source or heat generation technology that is ultimately implemented/used at the end-user levels depends on a number of factors. Among these, geographical and spatial characteristics (e.g., availability of heat network infrastructure, building construction features, outdoor temperature, etc.), in particular, play a crucial role. Su et al. \cite{su2018heating} deal in their study with optimal sustainable heating system alternatives with a special focus on local geographical features of the application site. Their results show that there might not be a one-fits-all solution if decarbonizing local heating systems. However, certain trends are very much emerging in their findings, which can also be confirmed by further case studies. Renewable-fed district heating networks have significant potential to supply heat demand in urban areas. This is exemplarily also shown by the results of Popovski et al. \cite{popovski2018technical}. They state that from a socio-economic perspective, district heating networks with excess heat are the most favorable supply option in densely populated areas. Lake et al. \cite{lake2017review} present a comprehensive review of district heating and cooling systems. They analyze among others the economic feasibility and system identification based on primary energy sources of centralized heating and cooling networks. Rama et al. \cite{rama2018introduction} study the optimal combination of different sustainable heating alternatives.In particular, they show how heat pumps and solarthermal can assist district heating networks. There exist also other alternatives. Sopha et al. \cite{sopha2011exploring} focus in their study on the potential of wood-pellet in Norway, a country with high shares of district heating-based heat supply. They use an agent-based model to identify energy policy options supporting the uptake of such sustainable heating systems. The authors conclude that a stable financial support (i.e., stable wood-pellet price) has the highest impact on the transition of wood-pellet. We refer to Section \ref{aspect3} in this context for a detailed discussion of financial incentives for renewable energy technologies in the heating sector.\vspace{0.5cm}

In any case, there is a need for sustainable alternatives to district heating. Either to complement existing district heating networks in a high-efficient way (e.g., \cite{rama2018introduction} and \cite{sopha2011exploring}) and/or because to compensate non-existing networks in the future. Popovski et al. \cite{popovski2018technical} identify the electrification of the heat supply using heat pumps with photovoltaics as the most cost-competitive alternative from a socio-economic perspective. Leibowicz et al. \cite{leibowicz2018optimal} also show end-use electrification as an optimal strategy for the decarbonization of the heating sector. However, the authors state that the electrification using heat pumps for example only makes sense in combination with building thermal efficiency improvements.\vspace{0.5cm}

In order to emphasize the importance of building renovation measures, we dedicate this concluding paragraph the corresponding literature. In particular, we select papers focusing on the impact of different retorfitting measures on sustainable heating system alternatives. However, we do not differentiate here in detail between different types of retrofitting measures (e.g., purely passive, passive, active, etc.) and refer in this context to the comprehensive literature review of Fina et al. in \cite{fina2019profitability}. Ma et al. \cite{ma2012existing} provide an extensive literature and state-of-the-art analysis of retrofitting focusing on existing buildings. Vieites et al. \cite{vieites2015european} elaborate in this context of European initiatives improving  the energy efficiency in existing and old (historic) buildings. Recently, Weinberger et al. \cite{weinberger2021investigating} investigate the impact of retrofitting on district heating networks. Fina et al. \cite{fina2019profitability} put their focus on the profitability of retrofitting of multi-apartment buildings with special consideration of different heating systems. They thoroughly study the implementation of the combination of building-attached/integrated photovoltaics supporting sustainable heating systems. Their results show how (passive) retrofitting measures result in a reduction of the required installed heating system capacity. However, the energy cost reduction achieved from higher building standards are not able to compensate the initial passive renovation investment costs. They conclude that latter significantly depend on the development of the CO\textsubscript{2} price and the assumptions of end-user investment grants as well as subsidies. We again take up these findings associated with financial support in Section \ref{aspect3}

\subsection{Justice in energy systems: fair and socially balanced sustainable energy transition}\label{aspect2}
The issue of justice in energy systems is addressed in various studies. According to them, a key part of achieving climate targets is to ensure that no one is left behind in the climate action. More generally, the three energy justice tenets are distributive, recognition, and procedural\footnote{In some works, restorative and cosmopolitan justice are also mentioned in this context. See, exemplarily in \cite{oxfordjustice2021}.}. Recently, these are comprehensively discussed and reviewed by Pellegrini et al. \cite{pellegrini2020energy}. Considering this work's scope, we put our focus on procedural justice, as it represents measures that reduce potential barriers to new clean energy investments \cite{oxfordjustice2021}.\vspace{0.5cm}

Generally speaking, dealing with just sustainable energy systems is a monumental task and seems to be very challenging to be generalized. However, studies focusing on certain local regions are likely to be the most promising approach. Recently, van Bommel and Höffken conducted a review study focusing on energy justice at the European community level \cite{van2021energy}. Besides that, Lacey-Barnacle et al. \cite{lacey2020energy} focus in their study on energy justice in developing countries. Coming back to this paper's content and spatial scope, Mundaca et al. \cite{mundaca2018successful} propose two local European case studies in Germany and Denmark investigating local energy transition from an energy justice perspective. Their findings are in line with those from Jenkins et al. \cite{jenkins2018humanizing} showing that energy justice and transitions framework can be combined and achieved simultaneously. However, Hiteva and Soacool \cite{hiteva2017harnessing} conclude from a business model perspective that energy justice may be realized through market principles but not through the market alone. We continue discussing this point in Section \ref{aspect3} when dealing with necessary (financial) incentives that foster the sustainable energy transition. \vspace{0.5cm} 

Recently, Hanke et al. \cite{hanke2021renewable} investigate renewable energy communities and their capability to deliver energy justice. They explore insights from 71 European cases and highlight the necessity of distributing affordable energy to vulnerable households. Furthermore, it is necessary to focus in this regard on low-income households. Exemplarily, Xu and Chen \cite{xu2019energy} propose on the basis of their generated results that low-income households need tailored assistance to ensure energy justice. In particular, they demonstrate that low-income households are renters and thus have fewer energy efficiency appliances. Sovacool et al. \cite{sovacool2019temporality} heat in the same direction and discuss the special difficulties for households without the capital for sustainable energy investments and for those that do not own their own home such as renters. Moreover, renters also often have higher residential heating energy use intensity, an energy efficiency proxy \cite{reames2016targeting}. In this context, Greene \cite{greene2011uncertainty} discussed the so-called “efficiency gap” or “energy paradox". He showed that consumers have a bias leading to undervaluation of future energy savings in relation to their expected value. The main reasons are a combination of two aspects, namely, an uncertainty regarding the net value of future fuel savings and the loss aversion of typical consumers. Filling the abovementioned efficiency gap is crucial in order to achieve both the energy transition and energy justice. Sovacool et al. \cite{sovacool2019decarbonization} show that unfolding the energy transition result in deeper injustices investigating four different low-carbon transitions.
%“They are grinding us into the ground” – The lived experience of (in)energy justice amongst low-income older households: Energy justice was experienced on four separately distinguishable levels of social relationships: intra-households, household-energy retailer relations, immediate social networks and wider social relations. simple retrofits improved householder heating capabilities \cite{willand2018they}

\subsection{Trade-offs between overnight investments and net present value decisions}\label{aspect3}
\todo{Wednesday - 27.10.2021}
\begin{itemize}
	
	\item[\textcolor{col}{\textbullet}] \textcolor{col}{Review paper welche Kriterien/Aspekte beeinflussen Investition in erneuerbare Energien}
	\item[\textcolor{col}{\textbullet}] \textcolor{col}{Finanzielle Unsicherheit der Hauptgrund warum nicht stärker investiert wird}
	\item[\textcolor{col}{\textbullet}] \textcolor{col}{neben investoren private investitionen sehr wichtig, damit gemeint auf lokaler ebene small-scale}
	\item[\textcolor{col}{\textbullet}] \textcolor{col}{Problem zum Beispiel Fernwärmeausbau in Gebiet profitabel für Unternehmen, aber nicht für Endkunden beispielsweise.}
	\item[\textcolor{col}{\textbullet}] \textcolor{col}{die meisten Arbeiten sprechen davon, dass öffentliche Anreize notwendig sind.}
	\item[\textcolor{col}{\textbullet}] \textcolor{col}{Effizienz wird so bestimmt, dass möglichst wenig eingegriffen werden soll}
	\item[\textcolor{col}{\textbullet}] \textcolor{col}{Einspeisetarif aber das bei Wärmesektor schwer}
	\item[\textcolor{col}{\textbullet}] \textcolor{col}{Studien Wärmesektor homeowners’ decision-making processes, 1-Familien und 2-Familien homeowners’ decision-making processes aber nicht Besitzverhältnisse auf Mehrparteienhäuser}
\end{itemize}

\begin{itemize}
	\item Ozorhon et al. (2018) \cite{ozorhon2018generating} Literaturübersicht welche Kriterien Investitionsentscheidung in erneuerbare Energien Technologien am stärksten beeinflussen. Allerdings wird allgemein von Investoren gesprochen und nicht explizit auf den privaten Sektor bzw. Staat fokussiert. 
	% main criteria that influence the decisions of the investors are determined based on an extensive literature survey and investigation of sector practice. 	
	\item Masini et al. (2012) \cite{masini2012impact} finanzielle Anreize sind der Hauptgrund warum erneuerbare Energien nicht stärker umgesetzt werden.	
	%	Yet, despite their appeal, and the numerous policies implemented to promote these technologies, the diffusion of RE projects remains somehow below expectations. This limited penetration is also due to a lack of appropriate financing and to a certain reluctance to invest in these technologies.
	\item Reuter et al. (2012) \cite{reuter2012renewable} allgemeinen überblick welche öffentlichen Anreize für Investitionen in erneuerbare Energie	
	% Einspeisetariffen hin zu Investitionszuschüssen, Steuergutschriften, Portfolioanforderungen und Zertifikatssystem	(Public incentives for companies to invest in renewable technologies range from feed-in tariffs, to investment subsidies, tax credits, portfolio requirements and certificate systems.) compare different policies to foster investment into renewables
	\item Zhou et al. (2011) \cite{zhou2011designing} effizienz einer Maßnahme nach dem notwendigen Maß an Eingriff
	% whereas the efficiency is defined as the amount of intervention, including taxes collected, subsidies paid, and GEP cost increase, to achieve the policy goal. The less intervention needed to achieve a goal, the higher the efficiency. proposed bilevel optimization model is to achieve a policy goal with a minimal amount of policy intervention  a centralized planner makes investment decisions for the energy system to serve projected demand of electricity.  berücksichtigt damit nicht die unterschiedlichen Besitzverhältnisse und Bedürfnisse, wir werden daher nicht die kostenminimale Lösung finden
	\item Couture et al. (2010) \cite{couture2010analysis}: bestätigt, dass feed-in tariffs are the most effective policy to encourage the rapid and sustained deployment of renewable energy. Stromsektor aber nicht Wärmesektor/Gebäudesektor
	% an overview of seven different ways to structure the remuneration of a FIT policy, divided into two broad categories: those in which remuneration is dependent on the electricity price, and those that remain independent from it.
	\item Hecher et al. (2017) \cite{hecher2017trigger} fokussiert auf homeowners’ decision-making processes, single and double-family houses. subsidies for heating system tabinvestments and infrastructural adjustments reveal to be most effective for homeowners in problem situations to foster alternative heating systems.
\end{itemize}

\begin{itemize}
	\item Wustenhagen et al. (2012) \cite{wustenhagen2012strategic} Wir brauchen private Investitionen in erneuerbare Energien
	%	Substantial private investment is needed if public policy objectives to increase the share of renewable energy and prevent dangerous anthropogenic climate change are to be achieved. 
	\item Eitan et al. (2019) \cite{eitan2019community} Review,  Beschreibt das so-called "phenomenon" of community and private sector renewable energy partnerships
	%	This review brings to the fore the fast-growing and significant phenomenon of community and private sector renewable energy partnerships, which constitute a fundamental building block of the global renewable energy transformation 
	\item Aslani et al. (2012) \cite{aslani2012prime} Stellt auf Basis einer umfassenden Literaturübersicht fest, dass private Sektor und dessen Investitionen sehr wichtig sind, fokussiert dabei sehr stark auf Stromsektor und kommt zu dem Schluss, dass der Staat als stärkster Treiber gesehen werden kann.
	%	This study will enrich the existing literature on renewable energy policy which emphasises the importance of engaging the private sector. investment in the Middle East 
	\item Rodriguez et al. (2015) \cite{rodriguez2015renewable} Effekte des Staates auf private Investitionen in erneuerbare Energien. Wieder Strom
	%	This paper analyses the effect of government policies and other determinants on private finance investment in renewable energy. A unique dataset of financial transactions for renewable energy projects is constructed using the Bloomberg New Energy Finance database. The dataset covers 87 countries, six renewable energy sectors (wind, solar, biomass, small hydropower, marine and geothermal) and the 2000–2011 time-span. In a first set of models undertaken at the level of the financial deal we find that, in contrast to quota-based schemes, price-based support schemes are positively correlated with private finance contributions.
	\item Schmidt et al. (2013) \cite{schmidt2013attracting} Selbst wenn der Wärmesektor untersucht wird, geht es oftmals um elektrifizierung
	% Attracting private investments into rural electrification — A case study on renewable energy based village grids in Indonesia private sector are essential to scale-up the diffusion. However, the findings also point to the need for government action in order to further improve the risk/return profile and thereby attract private investments for RVGs. 
	\item Williams et al. (2015) \cite{williams2015enabling} Barrieren für Teilnahme des privaten Sektors an dezentraler Elektrifizierungsprojekten
	%	The purpose of this paper is to review barriers to private sector participation in decentralized electrification projects and to identify solutions that have been implemented and proposed to overcome these barriers. The barriers discussed include unsecure revenue streams, inability to finance projects, and long-term project risks such as grid encroachment. The range of interventions and business models reviewed include methods to secure reliable demand, subsidy models, risk guarantees, and different revenue models. 
	\item Ostergaard et al. \cite{ostergaard2019costs} Investments in heating systems are attractive from an energy system perspective, Customer investments in heating systems should be motivated economically.
	\item Nageli et al. (2020) \cite{nageli2020policies} increase in the CO2 tax as well as subsidies are effective in speeding up the transition in the beginning, aber eben auch hier nicht die unterschiedlichen Besitzverhältnisse 
\end{itemize}

\subsection{Progress beyond state-of-the-art}
\todo{Thursday - 28.10.2021}
\begin{itemize}
	\item heating system change and sustainable heat supply takes place \textcolor{red}{justice in the heating sector}
	\item analytical and modeling framework; justice; qualitative analysis
	\item Trade-off analysis between governance, landlords, and tenants $\longrightarrow$ required incentives
	\item Sensitivity analysis helps us to bettter understand the different ownership structure and its influence on the sustainable energy transition
\end{itemize}