\section{State-of-the-art and progress beyond}\label{stateoftheart}

Focus here on renewable technologies in the heating sector; without energy communities; sharing renewable energy generation

\subsection{Local renewable heating and cooling systems}
\begin{itemize}
	\item options for the sustainable provision of heating and cooling demand: \cite{seyboth2008recognising}, \cite{popovski2018technical}
	\item using network infrastructure (district heating and cooling) $\longrightarrow$ fueling energy mix: \cite{lake2017review}: primary energy sources
	\item local on-site generation
\end{itemize}

\subsection{Justice in energy systems: fair and socially balanced sustainable energy transition}
the three energy justice tenets (distributive, recognitional and procedural)


A review Energy justice - equality \cite{pellegrini2020energy}
Sehr schwer aber zu generalisieren weil geopgraphische und lokale Apsekte berücksichtigt werden sollten
Energy justice in the transition to low carbon energy systems: Exploring key themes in interdisciplinary research: non-western 
Westliche Welt nicht auf die ganze Welt schließen (inklusive nicht entwickelte Länder) \cite{broto2018energy} case study von Mozambik

Harnessing social innovation for energy justice: A business model perspective: Energy justice may be operationalised through market principles but not through the market alone \cite{hiteva2017harnessing}

We examine two local energy transitions from an energy justice perspective \cite{mundaca2018successful}

show that energy justice and transitions frameworks can be combined \cite{jenkins2018humanizing}

Recently 2021, do renewable energy communities deliver energy justice? Exploring insights from 71 European cases \cite{hanke2021renewable}

\begin{itemize}
	\item "We aim to show how when low-carbon transitions unfold, deeper injustices related to equity, distribution, and fairness invariably arise" \cite{sovacool2019decarbonization}
	\item energy justice in household low carbon innovations; low carbon heating; collection of opportunities but also threats; who wins and who loses; difficulties to people without the capital, or who do not own their own home. (\cite{sovacool2019temporality})
\end{itemize}

Wir müssen speziell auf low-income households und renters schauen: Energy efficiency and energy justice for U.S. low-income households: An analysis of multifaceted challenges and potential: Low-income households and renters have fewer energy efficiency appliances; need tailored assistance \cite{xu2019energy}

low-income haben auch höheren residential heating energy use intensity, an energy efficiency proxy \cite{reames2016targeting}

“They are grinding us into the ground” – The lived experience of (in)energy justice amongst low-income older households: Energy justice was experienced on four separately distinguishable levels of social relationships: intra-households, household-energy retailer relations, immediate social networks and wider social relations. simple retrofits improved householder heating capabilities \cite{willand2018they}

\subsection{Trade-offs between overnight investments and net present value decisions}

\subsection{Progress beyond state-of-the-art}
\begin{itemize}
	\item heating system change and sustainable heat supply takes place
	\item analytical and modeling framework; justice; qualitative analysis
	\item Trade-off analysis between governance, landlords, and tenants $\longrightarrow$ required incentives
	\item Sensitivity analysis helps us to bettter understand the different ownership structure and its influence on the sustainable energy transition
\end{itemize}