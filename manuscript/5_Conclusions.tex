\section{Conclusions and recommendations}\label{conclusions}


sozial ausgewogenen bedeutet hier, dass kosten für den mieter dürfen nicht stark ansteigen 
investitionsförderung und dem mieter erlauben höhere mieten zu verlangen aufgrund von renewable enery systems


viel mehr muss der staat hier geld in die hand nehmen um profitability sicherzustellen 

Socially-balanced heating system transition in urban areas (landlords and tenants) hilft uns CO2 nicht als Trigger anders als in übrigen analysen

co2 nur als trigger wenn man die kosten auf des co2 preises auf beide akteure aufteilt und nicht nur auf den mieter packt, dann kann co2 preis trigger sein, dass hat die sensitivität analyse gezeigt

zeigt, dass wir uns nicht nur alleine auf den co2 verlassen dürfen, im wärmesektor muss geld in die hand genommen werden. 


% TODO: Future work nur wenn bei Gebäudesanierung und Wärmedämmung was zu einer Erweiterung des Modells führen würden (Stichwort Klimaneutralität 2050).
Future work: investment grants depend on buidling renovation