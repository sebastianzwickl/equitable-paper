\section{Conclusions and recommendations}\label{conclusions}


sozial ausgewogenen bedeutet hier, dass kosten für den mieter dürfen nicht stark ansteigen 
investitionsförderung und dem mieter erlauben höhere mieten zu verlangen aufgrund von renewable enery systems


viel mehr muss der staat hier geld in die hand nehmen um profitability sicherzustellen 

Socially-balanced heating system transition in urban areas (landlords and tenants) hilft uns CO2 nicht als Trigger anders als in übrigen analysen

co2 nur als trigger wenn man die kosten auf des co2 preises auf beide akteure aufteilt und nicht nur auf den mieter packt, dann kann co2 preis trigger sein, dass hat die sensitivität analyse gezeigt

zeigt, dass wir uns nicht nur alleine auf den co2 verlassen dürfen, im wärmesektor muss geld in die hand genommen werden. 


%Sustainable energy transition requires methods to bridge the gap between global decarbonization pathways and the resulting necessary measures at a local level. This work emphasizes the development of different downscaling algorithms, which we apply to the Austrian heating sector (residential and commercial) under several storylines in line with the Paris Agreement. We analyse results at the community and grid levels, considering technology-specific infrastructure requirements for the highly efficient usage of heat sources.\vspace{0.3cm}
%
%We found that the prioritized perspective of efficiency and local utilization of renewable heat sources implies substantial changes for the further development of district heating networks in the decarbonized Austrian heat supply toward 2050. This implies small-scale ($<\SI{1}{TWh}$) and large-scale ($>\SI{12}{TWh}$) district heating networks in terms of the amount of heat delivered. The results demonstrate that particularly densely populated areas are still beneficial supply areas for district heating networks and offer adequate heat densities. Nevertheless, most district heating networks in 2050 (seven of eight) will not reach the heat density benchmarks of today's networks and have a significant heat density gap. However, considering the increasing importance of local renewable heat sources feeding into district heating networks, we assume that these centralized networks will become required in the future and crucial in the decarbonization of the heating sector.\vspace{0.3cm}
%
%We anticipate our work as a starting point for discussing the role of centralized heat network infrastructure for enabling large-scale, highly efficient and local integration of renewable heat sources such as biomass/waste, hydrogen, ground-sourced heat pumps, or geothermal units. In particular, we see a need for further research on the trade-off between local integration of heat sources and the cost-intensive deployment of district heating networks. Future work may elaborate on the increasing cooling demand and how the cooperative design of district heating and cooling networks can contribute to the profitability of centralized heating and cooling infrastructure.

Future work: investment grants depend on buidling renovation