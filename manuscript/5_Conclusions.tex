\section{Conclusions and recommendations}\label{conclusions}
Rapid and equitable decarbonization of the building heat sector is an indispensable cornerstone in a sustainable society. Special attention is needed for the rented residential buildings sector since a sustainable investment decision is in the landlord's hands. Simultaneously, an expected increase in the CO\textsubscript{2} price primarily impacts the tenant's energy costs. This work studies cost-optimal federal subsidy payment strategies incentivizing sustainable heat change and retrofitting measures at the multi-apartment building level. We analyze the results of a partly renovated old building connecting to the district heating network and implementing an air-sourced heat pump system under several decarbonization storylines.\vspace{0.5cm}

We found that a fair sustainable heat system change is possible but with massive federal subsidy payments. In particular, the building's owner investment grant and additional rent-related revenues based on the building modernization are crucial to trigger the profitability of the investment. At the same time, subsidy payments are required at the beginning of the investment period to limit the energy and rent-related spendings of the tenants. Furthermore, the results imply that the heat pump alternative is not competitive in supplying heat service needs in partly renovated old buildings. Either the subsidy payments are significantly higher than in the district heating case, or the equitable constraints of the model can not be satisfied. Building renovation and reducing heat demand lead to feasibility but with high total costs because passive retrofitting measures need to be incentivized.\vspace{0.5cm}

Moreover, the results demonstrate that allocating the costs of inaction between the governance, the building owner, and the tenants is an important lever and can reduce the required subsidy payments. First and foremost, the biggest drop of the objective value (to nearly half) takes place when the costs of inaction are completely borne by the building owner. Also, a decrease in the landlord's interest rate reduces the total costs but limits the maximum share of the costs of inaction allocated to the landlord and implies a lower bound of the cost-minimized solution.\vspace{0.5cm}

Future work may investigate a stronger coupling of active and passive renovation measures as a necessary condition for federal subsidy payments. This could bring further insights to decarbonization strategies with an eye on the heat demand and sustainable heat source alternatives in the residential building sector (i.e., climate neutrality in 2050). Besides, the tenant's set-up of the building could be improved. In particular, further work should include different types of tenants within the building (e.g., different willingness to pay). More generally, this study could be extended by introducing further technology options, such as solar photovoltaic, solar thermal, and heat and electricity storage systems. 